\section{Data Structures}

\subsection{DSU (Disjoint Set Union)}
\begin{lstlisting}
// DSU with path compression and union by rank
class DSU {
private:
    vector<int> parent, rank;
public:
    DSU(int n) {
        parent.resize(n+1);
        rank.resize(n+1, 0);
        for (int i = 0; i <= n; i++) parent[i] = i;
    }

    int find(int x) {
        if (parent[x] != x) parent[x] = find(parent[x]);
        return parent[x];
    }

    void unite(int x, int y) {
        x = find(x), y = find(y);
        if (x == y) return;
        if (rank[x] < rank[y]) swap(x, y);
        parent[y] = x;
        if (rank[x] == rank[y]) rank[x]++;
    }

    bool same(int x, int y) {
        return find(x) == find(y);
    }
};
\end{lstlisting}

\subsection{Segment Tree}
\begin{lstlisting}
// Segment Tree for range queries and updates
class SegmentTree {
private:
    vector<ll> tree;
    int n;

    void build(vector<ll> &arr, int node, int start, int end) {
        if (start == end) {
            tree[node] = arr[start];
        } else {
            int mid = (start + end) / 2;
            build(arr, 2*node, start, mid);
            build(arr, 2*node+1, mid+1, end);
            tree[node] = tree[2*node] + tree[2*node+1];  // Change operation as needed
        }
    }

    void update(int node, int start, int end, int idx, ll val) {
        if (start == end) {
            tree[node] = val;
        } else {
            int mid = (start + end) / 2;
            if (idx <= mid) update(2*node, start, mid, idx, val);
            else update(2*node+1, mid+1, end, idx, val);
            tree[node] = tree[2*node] + tree[2*node+1];
        }
    }

    ll query(int node, int start, int end, int l, int r) {
        if (r < start || l > end) return 0;  // Change identity element as needed
        if (l <= start && end <= r) return tree[node];
        int mid = (start + end) / 2;
        return query(2*node, start, mid, l, r) + query(2*node+1, mid+1, end, l, r);
    }

public:
    SegmentTree(vector<ll> &arr) {
        n = arr.size();
        tree.resize(4*n);
        build(arr, 1, 0, n-1);
    }

    void update(int idx, ll val) {
        update(1, 0, n-1, idx, val);
    }

    ll query(int l, int r) {
        return query(1, 0, n-1, l, r);
    }
};
\end{lstlisting}

\subsection{Fenwick Tree (BIT)}
\begin{lstlisting}
// Fenwick Tree (Binary Indexed Tree)
class FenwickTree {
private:
    vector<ll> tree;
    int n;

public:
    FenwickTree(int size) {
        n = size;
        tree.resize(n+1, 0);
    }

    void update(int idx, ll delta) {
        for (; idx <= n; idx += idx & -idx) {
            tree[idx] += delta;
        }
    }

    ll query(int idx) {
        ll sum = 0;
        for (; idx > 0; idx -= idx & -idx) {
            sum += tree[idx];
        }
        return sum;
    }

    ll rangeQuery(int l, int r) {
        return query(r) - query(l-1);
    }
};
\end{lstlisting}

\subsection{Trie}
\begin{lstlisting}
// Trie (Prefix Tree)
class Trie {
private:
    struct Node {
        vector<Node*> children;
        bool isEnd;
        Node() : children(26, nullptr), isEnd(false) {}
    };
    Node* root;

public:
    Trie() { root = new Node(); }

    void insert(string word) {
        Node* curr = root;
        for (char c : word) {
            int idx = c - 'a';
            if (!curr->children[idx]) {
                curr->children[idx] = new Node();
            }
            curr = curr->children[idx];
        }
        curr->isEnd = true;
    }

    bool search(string word) {
        Node* curr = root;
        for (char c : word) {
            int idx = c - 'a';
            if (!curr->children[idx]) return false;
            curr = curr->children[idx];
        }
        return curr->isEnd;
    }

    bool startsWith(string prefix) {
        Node* curr = root;
        for (char c : prefix) {
            int idx = c - 'a';
            if (!curr->children[idx]) return false;
            curr = curr->children[idx];
        }
        return true;
    }
};
\end{lstlisting}

\subsection{Sparse Table}
\begin{lstlisting}
// Sparse Table for RMQ (Range Minimum Query)
class SparseTable {
private:
    vector<vector<ll>> table;
    vector<int> log;

public:
    SparseTable(vector<ll> &arr) {
        int n = arr.size();
        int maxLog = log2(n) + 1;
        table.assign(n, vector<ll>(maxLog));
        log.resize(n+1);

        for (int i = 0; i < n; i++) table[i][0] = arr[i];

        for (int j = 1; j < maxLog; j++) {
            for (int i = 0; i + (1 << j) <= n; i++) {
                table[i][j] = min(table[i][j-1], table[i + (1 << (j-1))][j-1]);
            }
        }

        for (int i = 2; i <= n; i++) log[i] = log[i/2] + 1;
    }

    ll query(int l, int r) {
        int j = log[r - l + 1];
        return min(table[l][j], table[r - (1 << j) + 1][j]);
    }
};
\end{lstlisting}

\subsection{Ordered Set (PBDS)}
\begin{lstlisting}
#include <ext/pb_ds/tree_policy.hpp>
#include <ext/pb_ds/assoc_container.hpp>
using namespace __gnu_pbds;

template<typename T> using ordered_set = tree<T, null_type, less<T>, rb_tree_tag, tree_order_statistics_node_update>;

// Usage:
ordered_set<int> os;
os.insert(5);
os.insert(2);
os.insert(7);
os.order_of_key(5);  // Returns number of elements < 5
*os.find_by_order(1);  // Returns element at index 1
\end{lstlisting}
